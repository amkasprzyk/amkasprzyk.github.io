\documentclass{article}
%------------------------------------------------------------------
\newcommand{\talk}[4]{\noindent $#1$pm \textbf{#2} (#3): \textit{#4}.}
\newcommand{\startabs}{\begin{center}\begin{tabular}{p{4.2in}}}
\newcommand{\finishabs}{\end{tabular}\end{center}\bigskip}
%------------------------------------------------------------------
\begin{document}
\thispagestyle{empty}
\begin{center}
\Large Calf Seminar
\end{center}
\bigskip
%------------------------------------------------------------------
 \talk{2.00}{Barrie Cooper}{University of Bath}{McKay Matrices, CFT
Graphs, and Koszul Duality (Part I)}
 \startabs
To a finite subgroup of $SL(n)$ we associate a graph. We explore
the possibility of classifying such graphs, and representation
theory highlights a recurrence relation for which these graphs
exhibit unusual behaviour. We reduce the qualitative behaviour
under this recurrence to a quantitative test, which the rational
conformal field theory graphs also appear to satisfy. In
subsequent talks we discuss how this test may betray the existence
of a pair of Koszul or almost-Koszul dual algebras associated to
the path algebra of the graph.
 \finishabs
 \talk{3.15}{Michal Kapustka}{University of Warwick}{Linear Systems on a K3 Surface}
 \startabs
The aim of this talk is to describe linear systems on K$3$
surfaces. We are mostly concerned with their base points (or
components), the morphism associated with them and its image. We
also try to introduce the notion of Seshadri constants and we show
some examples of linear systems on some K$3$ surface where we can
compute them.
 \finishabs
 \talk{4.30}{Grzegorz Kapustka}{University of Warwick}{Linear Systems on an Enriques Surface}
 \startabs
The aim of the talk is to describe linear systems of an
irreducible curve on an Enriques surface, and the maps associated
with these linear systems. We ask when a map is a morphism, what
is the degree of this morphism, and describe the eventual
singularities by looking at the image.
 \finishabs
%------------------------------------------------------------------
\end{document}
